\documentclass[14pt, a4paper]{article}
\usepackage[russian]{babel}
\usepackage{graphicx}
\usepackage{layout}
\usepackage[14pt]{extsizes}
\usepackage{amssymb}
\usepackage{relsize}

\setcounter{tocdepth}{4}
\setcounter{secnumdepth}{4}

\usepackage{xcolor}
\usepackage{hyperref}

\usepackage{listings}

 % Цвета для гиперссылок
\definecolor{linkcolor}{HTML}{000000} % цвет ссылок
\definecolor{urlcolor}{HTML}{000000} %цвет гиперссылок

\hypersetup{pdfstartview=FitH,  linkcolor=linkcolor,urlcolor=urlcolor, colorlinks=true}

\definecolor{codegreen}{rgb}{0,0.6,0}
\definecolor{codegray}{rgb}{0.5,0.5,0.5}
\definecolor{codepurple}{rgb}{0.58,0,0.82}
\definecolor{backcolour}{rgb}{0.97,0.97,0.97}

%таблица
\lstdefinestyle{mystyle}{
    backgroundcolor=\color{backcolour},   
    commentstyle=\color{codegreen},
    keywordstyle=\color{magenta},
    numberstyle=\tiny\color{codegray},
    stringstyle=\color{codepurple},
    basicstyle=\ttfamily\footnotesize,
    breakatwhitespace=false,         
    breaklines=true,                 
    captionpos=b,                    
    keepspaces=true,
    frame=single,                                                    
    showspaces=false,                
    showstringspaces=false,
    showtabs=false,                  
    tabsize=2,
    mathescape=true
}

\lstset{style=mystyle}

%Разметка страницы
\oddsidemargin = 0pt
\marginparwidth = 45pt 
\textwidth = 467pt
\textheight = 716pt
\topmargin = 0pt 
\footskip = 30pt 
\headheight = 0pt 
\headsep = 0pt 

\begin{document}
\begin{titlepage}
    \topmargin=216pt
    \newpage
    \hangindent=0.7cm
    \huge ИУ-10\\
    Системное\\
    Программное\\
    Обеспечение\\
    Администрирование Linux\\
    \textbf{Работа с сетью в Linux}

    \vspace{10cm}

    \begin{center}
        \small\textit{Москва, 2022}
    \end{center}
\end{titlepage}

\section*{На этом уроке} 
\addcontentsline{toc}{section}{На этом уроке}

\begin{enumerate}
    \item Обзор настроек, работающих на данный момент
    \item Обзор утилит для получения информации о сети (ip, ifconfig)
    \item Настройка сети в ОС Linux
    \item Сбор сетевого дампа и его анализ
\end{enumerate}

\tableofcontents
\newpage

\section*{Основы сети} 
\addcontentsline{toc}{section}{Основы сети}

\subsection*{IP адрес} 
\addcontentsline{toc}{subsection}{IP адрес}

Для настройки сети серверу нужен уникальный адрес и для этого используются IP-адреса. В
настоящее время актуальны две версии:
\begin{itemize}
    \item[-] Адреса IPv4: основаны на 32-битных значениях и имеют четыре октета, разделенных точками,
    например \underbar{\textit{192.168.10.100}}.
    \item[-] Адреса IPv6: основаны на 128-битных значениях и записываются в восьми группах
    шестнадцатеричных чисел по 16 бит каждая, разделенн- 
    ых двоеточиями. Адрес IPv6 может выглядеть как:\\
    \underbar{\textit{fe80:badb:abe01:45bc:34ad:1313:6723:8798}}.
\end{itemize}

Набор IP-адресов объединяют в сеть, в пределах которой компьютеры взаимодействуют почти
напрямую. Для связи между сетями используется маршрутизатор. Маршрутизатор - это машина (часто
для этой цели создано специальное оборудование), которая соединяет сети друг с другом.\\

Чтобы узнать, к какой сети принадлежит компьютер, для каждого IP-адреса используется маска
подсети. Маска подсети определяет, какая часть сетевого адреса указывает на сеть, а какая - на узел
(сервер, компьютер).

\subsection*{MAC-адрес} 
\addcontentsline{toc}{subsection}{MAC-адрес}

IP-адреса - это адреса, которые позволяют узлам связываться с любым другим узлом в Интернете.
Однако это не единственные используемые адреса. Каждая сетевая карта также имеет 6-байтовый
MAC-адрес, предназначенный для использования в локальной сети (вплоть до первого
обнаруженного маршрутизатора); они не могут использоваться для связи между узлами,
находящимися в разных сетях. MAC-адреса важны, потому что они помогают компьютерам найти
конкретную сетевую карту, которой принадлежит IP-адрес.\\

Примером MAC-адреса является \underbar{\textit{00:0c:29:7d:9b:17}}. Обратите внимание, что каждый MAC-адрес
состоит из двух частей. Первая половина - это идентификатор поставщика, а вторая - уникальный
идентификатор узла. Идентификаторы поставщиков зарегистрированы, и, используя
зарегистрированные идентификаторы поставщиков, можно выделить уникальные MAC-адреса.

\subsection*{Порт} 
\addcontentsline{toc}{subsection}{Порт}

На серверах Вы обычно будете запускать службы, такие как Веб-сервер или TFTP-сервер. Для
идентификации этих служб используются адреса портов. У каждой службы есть определенный адрес
порта, например порт 80 для протокола HTTP или порт 22 для Secure Shell (SSH), и при сетевой связи
отправитель и получатель используют адреса портов. Таким образом, существует адрес порта
назначения, а также адрес порта источника, участвующие в сетевых коммуникациях.

\section*{Управление конфигурацией сети} 
\addcontentsline{toc}{section}{Управление конфигурацией сети}

Сетевые адреса можно назначить двумя способами:
\begin{itemize}
    \item[-] Фиксированные IP-адреса: полезно для серверов, которые всегда должны быть доступны по
    одному и тому же IP-адресу.
    \item[-] Динамически назначаемые IP-адреса: полезно для устройств конечных пользователей и для
    экземпляров в облачной среде. Для динамического назначения IP-адресов обычно
    используется сервер протокола динамической конфигурации хоста (DHCP). 
\end{itemize}

\subsection*{Просмотр состояния сетевого стека} 
\addcontentsline{toc}{subsection}{Просмотр состояния сетевого стека}

\subsubsection*{Параметры интерфейсов} 
\addcontentsline{toc}{subsubsection}{Параметры интерфейсов}

Прежде чем вы настраивать сеть, вы должны знать, как проверить текущую конфигурацию. Полезной
информацией является:
\begin{itemize}
    \item[-] IP-адрес и маска подсети
    \item[-] Маршрутизация
    \item[-] Наличие портов и сервисов 
\end{itemize}

Чтобы проверить настройки сети, необходимо использовать утилиту ip. Утилита ip - это современная
утилита, с помощью которой можно отслеживать многие аспекты сети.
\begin{itemize}
    \item[-] ip addr для настройки и мониторинга сетевых адресов.
    \item[-] ip route для настройки и отслеживания информации о маршрутизации.
    \item[-] IP-ссылку для настройки и мониторинга состояния сетевого соединения. 
\end{itemize}

\newpage

Рассмотрим вывод команды \colorbox{backcolour}{ip addr show}

\vspace{0.3cm}
\begin{lstlisting}
root@server:$\sim$# ip addr show

1: lo: <LOOPBACK,UP,LOWER_UP> mtu 65536 qdisc noqueue state UNKNOWN group
default qlen 1000

    link/loopback 00:00:00:00:00:00 brd 00:00:00:00:00:00
    
    inet 127.0.0.1/8 scope host lo

       valid_lft forever preferred_lft forever
       
    inet6 ::1/128 scope host

       valid_lft forever preferred_lft forever

2: enp0s3: <BROADCAST,MULTICAST,UP,LOWER_UP> mtu 1500 qdisc fq_codel state 
UP group default qlen 1000

       link/ether 08:00:27:05:9e:a6 brd ff:ff:ff:ff:ff:ff

       inet 192.168.1.84/24 brd 192.168.1.255 scope global dynamic enp0s3

          valid_lft 83818sec preferred_lft 83818sec

       inet6 fe80::a00:27ff:fe05:9ea6/64 scope link

          valid_lft forever preferred_lft forever
$$
\end{lstlisting}
\vspace{0.2cm}


В результате этой команды можно ознакомиться со списком всех сетевых интерфейсов в Linux.
Обычно существуют как минимум два интерфейса, но в определенных случаях интерфейсов может
быть намного больше. Некоторые процессы используют протокол IP для внутренней связи. По этой
причине вы всегда найдете интерфейс \textbf{loopback}, с адресом 127.0.0.1. Важная часть вывода команды
относится к встроенной карте Ethernet. Команда показывает следующие элементы:

\begin{itemize}
    \item[-] Текущее состояние: наиболее важной частью этой строки является \textbf{UP}, которое показывает,
    что эта сетевая карта в настоящее время включена и доступна.
    \item[-] Конфигурация MAC-адреса: это уникальный MAC-адрес, который устанавливается для каждой
    сетевой карты. Вы можете увидеть сам MAC-адрес (08:00:27:05:9e:a6), а также
    соответствующий широковещательный адрес.
    \item[-] Конфигурация IPv4: в этой строке отображается текущий установленный IP-адрес, а также
    используемая маска подсети. Вы также можете увидеть широковещательный адрес, который
    используется для этой конфигурации сети. Обратите внимание, что на некоторых интерфейсах
    вы можете найти несколько адресов IPv4. \colorbox{backcolour}{dynamic} сигнализирует об использовании
    протокола DHCP.
    \item[-] Конфигурация IPv6: в этой строке отображается текущий адрес IPv6 и его конфигурация. Даже
    если вы ничего не настроили, каждый интерфейс автоматически получает адрес IPv6, который
    можно использовать для связи в локальной сети.
\end{itemize}

Команда \colorbox{backcolour}{ip link show} в сочетании с параметром -s, покажет текущую статистику о переданных и
полученных пакетах, а также обзор ошибок, возникших во время передачи пакетов.

\vspace{0.3cm}
\begin{lstlisting}
root@server:$\sim$# ip -s link show

1: lo: <LOOPBACK,UP,LOWER_UP> mtu 65536 qdisc noqueue state UNKNOWN mode
DEFAULT group default qlen 1000

    link/loopback 00:00:00:00:00:00 brd 00:00:00:00:00:00

    RX: bytes packets errors dropped overrun mcast

    13674      1430     0      0       0       0

    TX: bytes packets errors dropped carrier collsns

    13674      1430     0      0       0       0

2: enp0s3: <BROADCAST,MULTICAST,UP,LOWER_UP> mtu 1500 qdisc fq_codel state 
UP mode DEFAULT group default qlen 1000

    link/ether 08:00:27:05:9e:a6 brd ff:ff:ff:ff:ff:ff

    RX: bytes packets errors dropped overrun mcast

    103290692 127189    0      33      0       1386

    TX: bytes packets errors dropped carrier collsns

    823135     6204     0      0       0        0
$$
\end{lstlisting}
\vspace{0.2cm}

\textbf{Примечание!} \underbar{\textit{Подобную информацию можно получить с помощью}}\\
\underbar{\textit{команды \colorbox{backcolour}{ifconfig}.}}\\

Один из важных аспектов сети - маршрутизация, которая требуется для связи между узлами в
отличающихся сетях. В каждой сети есть маршрутизатор по умолчанию (также называемый шлюзом),
и вы можете увидеть, какой маршрутизатор используется в качестве маршрутизатора по умолчанию,
используя команду \colorbox{backcolour}{ip route show}.

\newpage

\begin{lstlisting}
root@server:$\sim$# ip route show

default via 192.168.1.254 dev enp0s3 proto dhcp src 192.168.1.84 metric 
100

192.168.1.0/24 dev enp0s3 proto kernel scope link src 192.168.1.84

192.168.1.254 dev enp0s3 proto dhcp scope link src 192.168.1.84 metric 100
$$
\end{lstlisting}

Самая важная часть - это первая строка. Она показывает, что маршрут по умолчанию проходит через
(“via”) IP-адрес \textbf{192.168.1.254}, а также показывает, что для адресации этого IP-адреса необходимо
использовать сетевой интерфейс \textbf{enp0s3}, и что этот маршрут был назначен DHCP-сервером. Метрика
используется в случае, если к одному месту назначения доступно несколько маршрутов. Маршрут с
наименьшей метрикойа предпочтителен.\\

Последующие строки идентифицируют локальные подключенные сети. При загрузке запись также
добавляется для каждой локальной сети, и в этом примере они относятся к сетям \textbf{192.168.1.0/24}. Эти
маршруты генерируются автоматически и не нуждаются в управлении.

\subsection*{Конфигурация сети} 
\addcontentsline{toc}{subsection}{Конфигурация сети}

С помощью утилит можно не только просматривать настройки но и изменять их. К примеру, \colorbox{backcolour}{ifconfig
eth0 192.168.0.100 netmask 255.255.255.0} назначит новый IP адрес, а \colorbox{backcolour}{route add default
gw 192.168.0.254} выставит шлюз по умолчанию. Все настройки, выполненные таким образом, будут
удалены в момент перезагрузки. Для “постоянной” конфигурации используется Netplan.

\subsubsection*{Netplan} 
\addcontentsline{toc}{subsubsection}{Netplan}

Netplan - это утилита для простой настройки сети в системе Linux. Описание требуемых сетевых
интерфейсов и того, как каждый из них должен работать, производится с помощью формата YAML. Yet
Another Markup Language (YAML) - «дружественный» формат сериализации данных, основное
внимание в котором необходимо обращать на отступы. Из этого описания Netplan генерирует всю
необходимую конфигурацию для выбранного вами инструмента рендеринга.\\

Конфигурация создается администраторами, установщиками или другими способами развертывания
ОС и находится в каталоге \textbf{/etc/netplan/}. Во время загрузки Netplan считывает содержимое файлов и
генерирует определенные конфигурацию серверной части в \textbf{/run}, чтобы передать управление
устройствами сетевому демону. Давайте рассмотрим конфигурацию нашего сервера:

\vspace{0.3cm}
\begin{lstlisting}
root@server:$\sim$# cat /etc/netplan/00-installer-config.yaml
# This is the network config written by 'subiquity'
network:
  ethernets:
    enp0s3:
      dhcp4: true
  version: 2
$$
\end{lstlisting}
\vspace{0.2cm}

\begin{itemize}
    \item[-] network — это блок начало конфигурации
    \item[-] renderer: networkd — сетевой менеджер, который будет использоваться для обработки
    (networkd или Network Manager)
    \item[-] version: 2 — версия формата YAML
    \item[-] ethernets: — блок говорит о том, что описываются ethernet интерфейсы.
    \item[-] enp0s3: — имя сетевого адаптера.
    \item[-] dhcp4: — состояние протокола DHCP
\end{itemize}

Как можно понять, на интерфейсах включен протокол DHCP. Использование этого протокола полезно
для устройств конечных пользователей и для экземпляров в облачной среде, но для серверных
версий рекомендуется использовать статическое назначение адресов.\\

\textbf{Важно!} \underbar{\textit{Перед началом изменения настроек сети рекомендуется сде-}} \\
\underbar{\textit{лать бэкап существующих файлов}}.\\

Изменим содержимое существующего файла, после чего применим с помощью \colorbox{backcolour}{netplan apply}.
Ответ на ввод команды будет получен только в случае наличия ошибок в файле.

\vspace{0.3cm}
\begin{lstlisting}
network:
    version: 2
    renderer: networkd
    ethernets:
        enp0s3:
            addresses:
                - 192.168.1.84/24
                - 192.168.1.87/24
            gateway4: 192.168.1.254
            nameservers:
                addresses:
                  - 192.168.1.254

root@server:$\sim$# netplan apply
root@server:$\sim$#
$$
\end{lstlisting}
\vspace{0.2cm}

С помощью опций \colorbox{backcolour}{addresses} и \colorbox{backcolour}{gateway4} указываются адреса сетевой карты и шлюза по
умолчанию. В блоке \colorbox{backcolour}{nameservers} отображены настройки DNS серверов. В \colorbox{backcolour}{ip addr show} исчезла
пометка \colorbox{backcolour}{dynamic}, и теперь на интерфейсе назначены сразу два адреса.

\vspace{0.3cm}
\begin{lstlisting}
2: enp0s3: <BROADCAST,MULTICAST,UP,LOWER_UP> mtu 1500 qdisc fq_codel state 
UP group default qlen 1000
    link/ether 08:00:27:05:9e:a6 brd ff:ff:ff:ff:ff:ff
    inet 192.168.1.84/24 brd 192.168.1.255 scope global enp0s3
        valid_lft forever preferred_lft forever
    inet 192.168.1.87/24 brd 192.168.1.255 scope global secondary enp0s3
        valid_lft forever preferred_lft forever
    inet6 fe80::a00:27ff:fe05:9ea6/64 scope link
        valid_lft forever preferred_lft forever
\end{lstlisting}
\vspace{0.2cm}

С помощью netplan можно по необходимости настраивать и маршрутизацию.

\vspace{0.3cm}
\begin{lstlisting}
root@server:$\sim$# cat /etc/netplan/00-installer-config.yaml
network:
    version: 2
    renderer: networkd
    ethernets:
        enp0s3:
            addresses:
                - 192.168.1.84/24
            gateway4: 192.168.1.254
            nameservers:
                addresses:
                  - 192.168.1.254
            routes:
                 - to: 8.8.8.8/32
                   via: 192.168.1.254
                   metric: 200

root@server:$\sim$# ip route show
default via 192.168.1.254 dev enp0s3 proto static
default via 192.168.1.254 dev enp0s3 proto static metric 200
192.168.1.0/24 dev enp0s3 proto kernel scope link src 192.168.1.84
\end{lstlisting}
\vspace{0.2cm}

После добавления маршрутов в выводе \colorbox{backcolour}{ip route show} появилась вторая запись с метрикой 200.\\

Генерируемые имена не всегда несут смысловую нагрузку. Для их изменения можно также
воспользоваться возможностями netplan. В файл понадобится добавить критерии соотношений
(\colorbox{backcolour}{match}) конфигурации с реальным устройство через MAC-адрес и указание имя в параметр
\colorbox{backcolour}{set-name}.

\vspace{0.3cm}
\begin{lstlisting}
root@server:$\sim$# cat /etc/netplan/00-installer-config.yaml
network:
    version: 2
    renderer: networkd
    ethernets:
        enp0s3:
            match:
                macaddress: 08:00:27:05:9e:a6
            set-name: eth0
            addresses:
                - 192.168.1.84/24
            gateway4: 192.168.1.254
            nameservers:
                addresses:
                  - 192.168.1.254
            routes:
                 - to: 8.8.8.8/0
                   via:
                   192.168.1.254
                   metric: 200
root@server:$\sim$# ip a show
1: lo: <LOOPBACK,UP,LOWER_UP> mtu 65536 qdisc noqueue state UNKNOWN group
default qlen 1000
    inet 127.0.0.1/8 scope host lo
        valid_lft forever preferred_lft forever
    inet6 ::1/128 scope host
        valid_lft forever preferred_lft forever
2: $\mathbf{eth0:}$ <BROADCAST,MULTICAST,UP,LOWER_UP> mtu 1500 qdisc fq_codel state 
UP group default qlen 1000
    link/ether 08:00:27:05:9e:a6 brd ff:ff:ff:ff:ff:ff
    inet 192.168.1.84/24 brd 192.168.1.255 scope global eth0
        valid_lft forever preferred_lft forever
    inet6 fe80::a00:27ff:fe05:9ea6/64 scope link
        valid_lft forever preferred_lft forever
$$
\end{lstlisting}
\vspace{0.2cm}

\textbf{Примечание!} \underbar{\textit{Для изменения имени понадобится перезагрузка сиc-}}\\
\underbar{\textit{темы}.}

\newpage

\section*{Проверка работы сети} 
\addcontentsline{toc}{section}{Проверка работы сети}

\subsection*{Ping} 
\addcontentsline{toc}{subsection}{Ping}

Ping - это служебная программа, используемая для проверки доступности хоста в сети по IP адресу.
Эта утилита доступна практически для всех операционных систем, имеющих сетевые возможности, в
том числе и Linux. Ping измеряет RTT(Round-Trip Time) сообщений, отправленных с исходного
компьютера на конечный хост.

\vspace{0.3cm}
\begin{lstlisting}
root@server:$$\sim# ping 8.8.8.8
PING 8.8.8.8 (8.8.8.8) 56(84) bytes of data.
64 bytes from 8.8.8.8: icmp_seq=1 ttl=109 time=22.6 ms
64 bytes from 8.8.8.8: icmp_seq=2 ttl=109 time=26.1 ms
64 bytes from 8.8.8.8: icmp_seq=3 ttl=109 time=23.9 ms
64 bytes from 8.8.8.8: icmp_seq=4 ttl=109 time=24.2 ms
^C
--- 8.8.8.8 ping statistics ---
4 packets transmitted, 4 received, 0% packet loss, time 3008ms
rtt min/avg/max/mdev = 22.641/24.218/26.094/1.236 ms
$$
\end{lstlisting}
\vspace{0.2cm}

По умолчанию утилита будет отправлять сообщения, пока не будет передана команда на прерывание.
С помощью ключа \textbf{-c} можно указать ограничение, а с помощью \textbf{-i} интервал.

\vspace{0.3cm}
\begin{lstlisting}
root@server:$\sim$# ping -i 0.1 -c 5 8.8.8.8
PING 8.8.8.8 (8.8.8.8) 56(84) bytes of data.
64 bytes from 8.8.8.8: icmp_seq=1 ttl=109 time=23.8 ms
64 bytes from 8.8.8.8: icmp_seq=2 ttl=109 time=23.0 ms
64 bytes from 8.8.8.8: icmp_seq=3 ttl=109 time=23.8 ms
64 bytes from 8.8.8.8: icmp_seq=4 ttl=109 time=23.6 ms
64 bytes from 8.8.8.8: icmp_seq=5 ttl=109 time=23.5 ms

--- 8.8.8.8 ping statistics ---
5 packets transmitted, 5 received, 0% packet loss, time 404ms
rtt min/avg/max/mdev = 23.025/23.528/23.780/0.272 ms
\end{lstlisting}
\vspace{0.2cm}

\subsection*{Порты приложений} 
\addcontentsline{toc}{subsection}{Порты приложений}

Сетевые проблемы могут быть связаны с локальным IP-адресом и настройками маршрутизатора, но
также могут быть связаны с сетевыми портами, которые недоступны на вашем сервере или на
удаленном сервере. Чтобы проверить доступность портов на вашем сервере, вы можете использовать
команду \colorbox{backcolour}{netstat} или более новую команду \colorbox{backcolour}{ss}, которая обеспечивает ту же функциональность.
\newpage

\begin{lstlisting}
root@server:$\sim$# netstat -tulpan
Active Internet connections (servers and
Proto Recv-Q Send-Q Local Address           Foreign               Stat
PID/Program name
tcp          0      0 127.0.0.53:53         0.0.0.0:*
             LISTEN
582/systemd-resolve
tcp          0      0 0.0.0.0:22            0.0.0.0:*
             LISTEN
633/sshd: /usr/sbin
tcp          0    0 192.168.1.84:22     192.168.1.73:60111
633/sshd: /usr/sbin
udp          0                0 127.0.0.53:53                  0.0.0.0:
582/systemd-resolve
udp6                   0                0 fe80::a00:27ff:fe05:546   :::*
580/systemd-network
root@server:$\sim$#
$$
\end{lstlisting}
\vspace{0.2cm}

Набрав \colorbox{backcolour}{ss -lt}, вы увидите все прослушивающие TCP-порты в локальной системе

\vspace{0.3cm}
\begin{lstlisting}
root@server:$\sim$# ss -lt
State               Recv-Q               Send-Q                     Local
Address:Port                            Peer                    Process
LISTEN                            0                                  4096
127.0.0.53%lo:domain                               0.0.0.0:
LISTEN                            0                                   128
0.0.0.0:ssh                                  0.0.0.0:
LISTEN                            0                                   128
[::]:ssh                                     [::]:


\end{lstlisting}
\vspace{0.2cm}

Обратите внимание, какие из портов прослушиваются. Некоторые порты прослушивают только адрес
loopback IPv4 127.0.0.1 или IPv6 :: 1, что означает, что они доступны только локально и недоступны с
внешних компьютеров. Другие порты прослушивают *, что означает все адреса IPv4, или ::: *, который
представляет все порты на всех адресах IPv6.

\subsubsection*{Netcat} 
\addcontentsline{toc}{subsubsection}{Netcat}

Для проверки портов на удаленном сервере следует воспользоваться утилитой Netcat и командой nc.
Netcat может осуществлять соединение по TCP портам, а также посылать и принимать TCP,UDP
трафик.\\

Для простой проверки доступности удаленного порта воспользуйтесь \colorbox{backcolour}{nc -zvw1}.

\newpage

\begin{lstlisting}
root@server:$\sim$# nc -zvw1 8.8.8.8 80
nc: connect to 8.8.8.8 port 80 (tcp) timed out: Operation now in progress
root@server:$\sim$# nc -zvw1 8.8.8.8 53
Connection to 8.8.8.8 53 port [tcp/domain] succeeded!
\end{lstlisting}
\vspace{0.2cm}

Чтобы послать послать UDP пакеты, воспользуйтесь опцией -u.

\vspace{0.3cm}
\begin{lstlisting}
root@server:$\sim$# root@server:$\sim$# nc -u 8.8.8.8 53
Sending some packet to GOOGLE DNS!
^C
\end{lstlisting}

\subsection*{Tcpdump} 
\addcontentsline{toc}{subsection}{Tcpdump}

Tcpdump является самой мощной утилитой для анализа работы сетевого стека. Эта утилита выводит
описание содержимого пакетов сетевого интерфейса.

\vspace{0.3cm}
\begin{lstlisting}
root@server:$\sim$# tcpdump
tcpdump: verbose output suppressed, use -v or -vv for full protocol decode
listening on eth0, link-type EN10MB (Ethernet), capture size 262144 bytes
13:43:42.366813   IP   server.ssh   >   192.168.1.73.60111:   Flags
                  seq
3538046563:3538046671,  ack 2391002831,  win 501,  options [nop,nop,TS val
1982379078 ecr 802292551], length 108
13:43:42.366891 IP server.ssh > 192.168.1.73.60111: Flags [P.], seq 
108:244, ack 1, win 501, options [nop,nop,TS val 1982379078 ecr 
802292551], length 136
13:43:42.367002 IP 192.168.1.73.60111 > server.ssh: Flags [.], ack 108, 
win 2046, options [nop,nop,TS val 802292585 ecr 1982379078], length 0
13:43:42.367002 IP 192.168.1.73.60111 > server.ssh: Flags [.], ack 244, 
win 2044, options [nop,nop,TS val 802292585 ecr 1982379078], length 0
13:43:42.367059 IP server.ssh > 192.168.1.73.60111: Flags [P.], seq 
244:272, ack 1, win 501, options [nop,nop,TS val 1982379079 ecr 
802292585], length 28
\end{lstlisting}
\vspace{0.2cm}

Описанию параметров пакета предшествует отметка времени, которая по умолчанию печатается в
виде часов, минут, секунд и долей секунды после полуночи.Работой tcpdump можно управлять с
помощью набора ключей и параметров.\\

В случае наличия нескольких интерфейсов с помощью ключа -i можно выбрать один, а с параметром
\colorbox{backcolour}{host} определить адрес, трафик до которого необходимо отобразить на экране.

\vspace{0.3cm}
\begin{lstlisting}
root@server:$\sim$# tcpdump -i eth0 host 8.8.8.8
tcpdump: verbose output suppressed, use -v or -vv for full protocol decode
listening on eth0, link-type EN10MB (Ethernet), capture size 262144 bytes
13:47:44.555289 IP server > dns.google: ICMP echo request, id 4, seq 356, 
length 64
\end{lstlisting}

\newpage

\begin{lstlisting}
13:47:44.578818 IP dns.google > server: ICMP echo reply, id 4, seq 356, 
length 64
13:47:45.558374 IP server > dns.google: ICMP echo request, id 4, seq 357, 
length 64
13:47:45.583456 IP dns.google > server: ICMP echo reply, id 4, seq 357, 
length 64
13:47:46.559604 IP server > dns.google: ICMP echo request, id 4, seq 358, 
length 64
13:47:46.581900 IP dns.google > server: ICMP echo reply, id 4, seq 358, 
length 64
13:47:47.561489 IP server > dns.google: ICMP echo request, id 4, seq 359, 
length 64
13:47:47.584801 IP dns.google > server: ICMP echo reply, id 4, seq 359, 
length 64

Src dst
\end{lstlisting}
\vspace{0.2cm}

Фильтрации по адресу источника и назначения производится с помощью параметров \colorbox{backcolour}{Src} и \colorbox{backcolour}{dst}
соответственно. В пример ниже видны только ответы от \textbf{8.8.8.8} , но не видно запросов.

\vspace{0.3cm}
\begin{lstlisting}
root@server:$\sim$# tcpdump -i eth0 src 8.8.8.8
tcpdump: verbose output suppressed, use -v or -vv for full protocol decode
listening on eth0, link-type EN10MB (Ethernet), capture size 262144 bytes
13:51:58.094222  IP dns.google > server: ICMP echo reply, id 4, seq 609,
64
13:51:59.099626  IP dns.googl  > server  ICMP echo reply, id 4, seq 610,
64
13:52:00.100698  IP dns.googl  > server  ICMP echo reply, id 4, seq 611,
64
$$
\end{lstlisting}
\vspace{0.2cm}

Описанные действия крайне удобны для сбора трафика, но в tcpdump существует возможность
комбинировать варианты, чтобы выделить именно те пакеты, которые требуются. Есть три способа
создания комбинаций, схожих с базовым программированием:
\begin{itemize}
    \item[-] И - \textbf{and} или \textbf{\&\&}
    \item[-] ИЛИ - \textbf{or} или \textbf{||}
    \item[-] Исключить - \textbf{not} или \textbf{!}
\end{itemize}

С помощью \colorbox{backcolour}{and} объединили условия адреса и порта.

\vspace{0.3cm}
\begin{lstlisting}
root@server:$\sim$# tcpdump -i eth0 host 8.8.8.8 and port 53
tcpdump: verbose output suppressed, use -v or -vv for full protocol decode
listening on eth0, link-type EN10MB (Ethernet), capture size 262144 bytes
13:54:09.475970 IP server.38878 > dns.google.domain: Flags [S], seq 
1882266456, win 64240, options [mss 1460,sackOK,TS val 2494499897 ecr 0,
nop,wscale 7], length 0
13:54:09.500842 IP dns.google.domain > server.38878: Flags [S.], seq 
1845975136, ack 1882266457, win 65535, options [mss 1430,sackOK,TS val 
341650861 ecr 2494499897,nop,wscale 8], length 0
13:54:09.500895 IP server.38878 > dns.google.domain: Flags [.], ack 1, win 
502, options [nop,nop,TS val 2494499921 ecr 341650861], length 0
13:54:11.523621 IP dns.google.domain > server.38878: Flags [F.], seq 1, 
ack 1, win 256, options [nop,nop,TS val 341652884 ecr 2494499921], length 
0
13:54:11.523843 IP server.38878 > dns.google.domain: Flags [F.], seq 1, 
ack 2, win 502, options [nop,nop,TS val 2494501944 ecr 341652884], length 
0
13:54:11.547083 IP dns.google.domain > server.38878: Flags [.], ack 2, win 
256, options [nop,nop,TS val 341652908 ecr 2494501944], length 0
^C
6 packets captured
6 packets received by filter
0 packets dropped by kernel
$$
\end{lstlisting}
\vspace{0.2cm}

С помощью исключения порта 22 собирается весь трафик, но без сессий SSH, которые часто только
мешают в анализе пакетов.

\vspace{0.3cm}
\begin{lstlisting}
root@server:$\sim$# tcpdump -i eth0 port ! 22
tcpdump: verbose output suppressed, use -v or -vv for full protocol decode
listening on eth0, link-type EN10MB (Ethernet), capture size 262144 bytes
14:12:11.118891 IP 192.168.1.75.50222 > 224.0.0.252.hostmon: UDP, length
22
14:12:11.119213   IP   server.42847   >   mygpon.domain:   25616+
                        [1au] PTR? 255.1.168.192.in-addr.arpa. (55)
14:12:11.126544 IP mygpon.domain > server.42847: 25616 NXDomain 0/1/1(104)
14:12:11.127434    IP     server.42847    >     mygpon.domain:    25616+
PTR? 255.1.168.192.in-addr.arpa. (44)
14:12:11.130657 IP mygpon.domain > server.42847: 25616 NXDomain 0/0/0 (44)
14:12:11.131321   IP   server.41849   >   mygpon.domain:   13992+   [1au]
                  PTR? 75.1.168.192.in-addr.arpa. (54)
14:12:11.138647 IP mygpon.domain > server.41849: 13992 NXDomain 0/1/1(103)
14:12:11.139065    IP     server.41849    >     mygpon.domain:    13992+
                   PTR? 75.1.168.192.in-addr.arpa. (43)
^C
6 packets captured
6 packets received by filter
0 packets dropped by kernel
$$
\end{lstlisting}
\vspace{0.2cm}

\textbf{Важно!} \underbar{\textit{С помощью опции -w <имя файла> трафик будет записан в}}\\
\underbar{\textit{указанный файл, а ключ -s0 уберет ограничение на размер записываемых}} 
\underbar{\textit{пакетов.}}

\newpage

\section*{Практическое задание} 
\addcontentsline{toc}{section}{Практическое задание}

\begin{enumerate}
    \item С помощью утилиты netplan повторить имеющиеся в системе параметры, но задать их
    статически.
    \item Запустить ping до сайта ya.ru, проверить доступность.
    \item Не останавливая ping, открыть отдельное окно терминала и вывести на экран только запросы
    в сторону (echo request) в сторону сайта ya.ru.
    \item С помощью утилиты nc проверить доступность основных портов (53,80, 443) у сайта
    ya.ru, habr.com, google.com .
    \item Проанализировать предложенный .pcap файл дампа сетевого трафика.
    \item Установить утилиту ssh, подключиться с хостовой машины к виртуальной машине по ssh
    (прим. необходимо сменить конфигурацию сетевого адаптера гостевой ОС). Отключить
    возможность аутентифицироваться пользователю root. Сделать аутентификацию с
    использованием публичного ключа.
\end{enumerate}

\newpage

\section*{Глоссарий} 
\addcontentsline{toc}{section}{Глоссарий}

\href{https://ru.wikipedia.org/wiki/IP-адрес}{\underbar{\textbf{IP-адрес}}} — уникальный сетевой адрес узла в компьютерной сети, построенной на основе стека
протоколов TCP/IP.\\

\noindent \href{https://ru.wikipedia.org/wiki/Порт_(компьютерные_сети)}{\underbar{\textbf{Порт}}} - сетевой идентификатор приложения.\\

\noindent \href{https://ru.wikipedia.org/wiki/MAC-адрес}{\underbar{\textbf{MAC адрес}}} - адрес сетевой карты.\\

\noindent \href{https://netplan.io/}{\underbar{\textbf{Netplan}}} - утилита для настройки сети в Linux.

\section*{Дополнительные материалы} 
\addcontentsline{toc}{section}{Дополнительные материалы}

\href{https://habr.com/ru/post/448400/}{\textit{Статья о netplan}}

\section*{Используемые источники} 
\addcontentsline{toc}{section}{Используемые источники}

\href{https://ru.wikipedia.org/wiki/Динамический_порт}{\textit{https://ru.wikipedia.org/wiki/Динамический\_порт}}\\
\href{https://netplan.io/examples/}{\textit{https://netplan.io/examples/}}

\end{document}